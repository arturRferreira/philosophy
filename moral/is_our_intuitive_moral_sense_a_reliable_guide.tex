\documentclass[a4paper,onecolumn,11pt]{article}


\usepackage[utf8]{inputenc}
\usepackage[portuguese]{babel}

\usepackage{indentfirst}
\PassOptionsToPackage{hyphens}{url}\usepackage{hyperref}

\usepackage[margin=3cm]{geometry}

% Title Page
\title{Is Our Intuitive Moral Sense a Reliable Guide?\\O Nosso Senso Intuitivo Moral é um Guia Confiável? }
\author{Arian Pasquali\\
	up201202551
	\and 
	Artur R Ferreira\\
	up200704289
	\and
	João Silva\\
	up200901628}

\begin{document}
\maketitle

\begin{abstract}
\end{abstract}


\section{Fundamental Tendencies Underlying the Human Moral System}

\section{Erros Concetuais no Sistema Moral Humano}
	\subsection{Divisão da Responsabilidade}
	Na presença de grandes grupos, é menos propício que as pessoas tomem a iniciativa de intervir ou se sintam menos responsáveis quando algo acontece. Este fenómeno designa-se \textbf{Divisão da Responsabilidade} e é frequentemente utilizado para explicar \textit{Efeito Espectador} \cite{kendra}, um fenómeno no qual quanto maior for o número de pessoas num grupo, menos provável é ajudar uma pessoa em dificuldades. Por exemplo, no meio de uma rua movimentada duma grande cidade, um homem cai ao chão e começa a ter convulsões. A maioria das pessoas olham para o homem, mas ninguém o ajuda ou chama assistência. Isto porque há tanta gente, que há a tendência de não se sentir pressionado a acudir.
	
	Em 1970, Darley and Latané fizeram uma experiência: um assalto numa loja de vinhos. Os ladrões eram dois jovens que entraram na loja e perguntaram por uma bebida cara importada. O funcionário da loja deslocou-se ao armazém para a buscar. Enquanto isto, os assaltantes agarraram uma caixa de cerveja que estava em cima do balcão e disseram um para o outro: "Não vão sentir a falta disto" e fugiram.
	O Objetivo da experiência \cite{pysch} era ver se os outros clientes relatavam o sucedido ao funcionário. Quando os fregueses estavam sozinhos, 65\% denunciaram. Em grupo apenas 56\%. 
	
	Caraterísticas pessoais da vítima foram identificadas como relevantes na taxa de intervenção dos sujeitos. 
	Rodin e Piliavin conduziram uma experiência na qual um colaborador colapsa num metropolitano. Numa versão da experiência, o homem estava mal vestido, barba grande e a cheirar a álcool. A maioria das pessoas não o ajudaram enquanto ele colapsava no chão do metro. Já um homem de negócios bem-vestido obteve mais respostas de ajuda perante a mesma experiência.
	
	Outros estudos mostraram que contato prévio encorajava o comportamento de ajuda perante o próximo mesmo que tenha sido um pequeno período de conversa.

	Jade Mazarin \cite{jade} identifica neste fenómeno quatro caraterísticas básicas.
	\begin{itemize}
		\item\textbf{ Falta de responsabilidade pessoal}  - Como referido anteriormente, é diferente quando estamos sozinhos comparado em grupo.
		\item \textbf{Diminuição do sentimento de culpa} - É mais fácil desculpabilizarmos por não ter ajudado o outro ou atribuir a culpa a outros caso haja mais pessoas.
		\item \textbf{Olhar para outros para ter um guia} - Pondera-se se alguem conhece melhor ou sabe lidar  com o que se está a passar melhor do que nós.
		\item \textbf{Tamanho do grupo} - Observou-se com o aumento do grupo, há a tendência de se sentir menos responsável por ajudar. Quanto mais pessoas puderem ajudar, maior o sentimento que não é preciso intervir.
	\end{itemize}

	Num último exemplo de divisão de responsabilidade, pode ser assinalado a desculpabilização dos oficiais Nazi julgados em Nuremberg. Todos alegaram que estavam a seguir ordens. Argumentaram que se o holocausto fosse tão mau como diziam, mais alguém deveria ter conhecimento da situação e que não era necessariamente a responsabilidade deles relatar às autoridades. Também argumentaram que as únicas autoridades em vigor na Europa Continental estavam sobre o controlo alemão e que seria suicido se tentassem avisar o mundo exterior, passando ser em vão a sua tentativa.
	Isto não é inteiramente verdade devido à maioria da população alemã não ter conhecimento disto, e caso o tivesse exigiria que parasse \cite{10cases}.

	

	\subsection{Imunidade Moral}
		Muitas decisões tomadas hoje em dia e que interferem no bem geral são decisões tomadas por grandes e poderosas instituições tal como os estados e corporações internacionais. Assim, aparenta ser difícil para as pessoas dividir a culpa com uma corporação. Thomas Nagel menciona que também se pode extender esta dificuldade aos cargos públicos.
	
		Crimes públicos são cometidos por indivíduos com papel na política, forças armadas e instituições económicas... Não são apenas soldados normais, polícias secretas mascarados na sua função, mas também secretários de estado ou até presidentes... Eles agem como funcionários públicos, como indivíduos separados de uma forma intrigante daquilo que fazem: isolados da sua própria visão e como os outros o observam. \cite{maxwell}

		Pode-se portanto dividir esta \textbf{imunidade moral} em dois subproblemas. O primeiro prende-se com as nações e grandes corporações serem consideradas pessoas, mas ainda assim não são aplicados os meus padrões morais como pessoas reais. Depois, os cargos públicos dessas instituições beneficiam duma certa imunidade de criticas morais. Se alguém perguntar quem fez o quê numa determinada situação, a resposta poderá ser "o estado" ou "a empresa", tal como se tratasse duma entidade com inteligência e vontade própria. No entanto, nenhum destes grupos se espera que tenham consciência ou aja como um simples indivíduo. De reparar que os maiores intervenientes no bem estar da população não sigam estes padrões morais. Isto porque também há a tendência de não serem responsabilizados pelo resto da sociedade pelos seus actos. 

		No caso do estado/nações, a sua isenção de moralidade pode estar relacionado com o tribalismo no qual o grupo em que pertence está sempre certo. Como precisamos da proteção do estado, especialmente contra um inimigo é nos obrigado que consideremos o nosso país bom. Se alguém criticar a forma de atuar, há sempre alguma desculpa a apresentar.

		Já no caso das empresas, a desobrigação da responsabilidade moral assenta no fato que é permitido pela sociedade ocupar a sua função. Essa função passa por organizar uma parte da economia. Uma empresa faz o que lhe compete quando obtém lucro, criando empregos, fornecendo o mercado. Como se poderia pedir o contrário? Cuidar dos feridos ou tornar a cidade sustentável e bonita não deverá ser o motivo principal das empresas existir embora possam contribuir significantemente para esse fato.
		
		Um caso célebre, já referenciado anteriormente, no qual juízes determinaram que os indivíduos eram responsáveis pelos seus atos mesmo quando sob ordens superiores foi o julgamento dos oficiais Nazi em 1945. Este caso tem contornos ainda mais especiais dado que uma nação foi julgada perante outra ao contrário de ser julgada num tribunal internacional.
		
		\subsection{Voluntarismo e Estruturalismo}
		
		O último erro conceptual a ser assinalado é um espelho dos últimos dois. A relação entre \textbf{voluntarismo} e \textbf{estruturalismo}. Por voluntarismo refere-se a crença geral que as pessoas são voluntárias e intervenientes morais responsáveis como sempre foi o padrão no sistema moral humano. Por estruturalismo entenda-se a sociedade interpreta eventos como causa não tanto pelos atos voluntários ou convicções pessoais mas por forças, entidades ou tendências inabaláveis. Assim, tal como referenciado anteriormente, ações efetuadas pelas empresas ou estados aparentam provir não da vontade geral mas do sistema que está por detrás das grandes instituições. Consequências serão explicadas por dinâmica ou lógica interna em vez escolhas humanas.
		
		O problema desta incompatibilidade é que não foi devidamente analisada. O sistema moral, tanto evoluído biologicamente como desenvolvido culturalmente, sempre foi um relevante árbitro na vida social humana. É preciso pegar em todas as atuações sociais na sua essência para ponderação da sua justiça e deve-se possuir mecanismos de recuperação de todos os intervenientes para que estes voltem ao caminho certo. Na situação atual, quantidades significativas da atividade social estão dispensadas de avaliação moral. Esta dispensa permite o crescimento de problemas sociais pondo em causa o futuro de algumas sociedades ou espécies. Este assunto também interfere na visão da moral por parte da sociedade, que começa a pensar como parte irrelevante da vida social.
		
		Em resumo, a convivência entre voluntarismo e estruturalismo é no mínimo complicada. A constante procura por reputação do voluntarismo por um papel relevante na vida humana, esconde o fato de que a crescente proporção da atividade foi retirada do controlo do voluntariado e até do escrutínio moral.
		
		
		
\bibliographystyle{plain}
\bibliography{relatorio1_biblio}

\end{document}          
